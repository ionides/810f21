\documentclass[12pt]{article}
\usepackage{fullpage,hyperref}\setlength{\parskip}{3mm}\setlength{\parindent}{0mm}
\begin{document}

\begin{center}\bf
Homework 14. Due by 5pm on Thursday 12/9.

A research seminar

\end{center}

Attending research seminars requires the skill of learning something meaningful from a presentation which will not all be readily comprehensible. An indication of active participation is to ask a relevant question. We are going to end the semester with a research seminar based on a paper to appear shortly in {\em Theoretical Population Biology}, available at \url{https://arxiv.org/abs/2105.12730}.

You will see that this paper contains elements of probabily, theoretical statistics, applied statistics, computational statistics, and science.  Although it may not be practical for you to fully understand the entire paper, there should be various points where you can make contact with the material at your current level of understanding of these topics.

Write brief answers to the following questions, by editing the tex file available at \url{https://github.com/ionides/810f21}, and submit the resulting pdf file via Canvas.

This class and homework are extra credit. As explained in an earlier class announcement, this makes allowance for the possibility that you may have had to miss one or two earlier classes for whatever reasons. Since the primary purpose of a PhD is to learn how to participate in the research community, hopefully this class will be worth your while even if you do not need the participation credit to get the course grade that you aspire to.

The last class will be run differently from before. I will present  from 9:00 until around 9:30, at which point there will be time for questions.

This paper was developed using the reproducible workflow investigated in Homework 13. If you are curious to see more, the article source code is at \url{https://ionides.github.io/810f21/mgp/ms.Rnw}. The code also uses an R package called phylopomp, which is  available on the \href{https://zenodo.org/record/5758900}{Zenodo repository} for the article at \url{https://zenodo.org/record/5758900/files/phylopomp_0.0.23.1.tar.gz}.

\begin{enumerate}

\item Read the article above and propose a question. This does not have to be the same as the question you ask in class, since other thoughts might arise later.
  
YOUR ANSWER HERE

\item A research seminar is usually expected to summarize years of technical research on a topic where the researchers spent even more years collecting expertise. If you go to seminars, you may feel you are wasting your valuable time listening to things you can only partially follow and don't critically need to know. On the other hand, if you do not go to seminars, you may feel you are missing a major opportunity to learn how research works. How do you balance these opposing views?

YOUR ANSWER HERE
  
\item Going to a seminar is a communal activity. It gives an opportunity to discuss what you think about the topic, and what you were able to learn from the presentation, with your peers. Have you yet had such an experience at a research seminar (the statistics department seminar, or elsewhere)? Explain briefly.

YOUR ANSWER HERE
    
\end{enumerate}
\end{document}
